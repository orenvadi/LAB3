\input{Preambule_course_work.tex}

\title{Графовые алгоритмы и структуры данных}
\author{Бактыбеков Н.Б.}

\begin{document}
\input{title/title.tex}

% \maketitle

\newpage
\tableofcontents{}
\setcounter{page}{1}

\newpage
\addcontentsline{toc}{section}{Индивидуальное задание}


\section*{Индивидуальное задание}
Решить две задачи:

\textbf{Задача 1}


Составить программу, которая содержит текущую информацию о книгах в библиотеке.
Сведения о книгах содержат:

\begin{compactitem}
    \item Номер удк;
    \item Фамилию и инициалы автора;
    \item Название;
    \item Год издания;
    \item Количество экземпляров данной книги в библиотеке.
\end{compactitem}

Программа должна обеспечивать:

\begin{compactitem}
    \item Начальное формирование данных о всех книгах в библиотеке в виде списка;
    \item При взятии каждой книги вводится номер УДК, и программа уменьшает 
    \item Значение количества книг на единицу или выдает сообщение о том, что 
    \item Требуемой книги в библиотеке нет, или требуемая книга находится на руках;
    \item При возвращении каждой книги вводится номер УДК, и программа 
    \item Увеличивает значение количества книг на единицу;
    \item По запросу выдаются сведения о наличии книг в библиотеке.
\end{compactitem}


\textbf{Задача 2}

Написать 4 представления графа.

\begin{compactitem}
    \item Матрица смежности;
    \item Матрица инцидентности;
    \item Списки смежности;
    \item Списки Рёбер.
\end{compactitem}

.А также реализивать алгоритм Дейкстры






\section{Задача 1}


\subsection{Реализация программы на python:}


\textbf{Пояснение программы:}

Для реализации данного задания на Python,
в коде были созданы 2 класса \textbf{Book} и \textbf{Library}
их методы и конструкторы классов.

А также ниже написан пример использования данной программы



\textbf{Код программы:}


\lstinputlisting[language = python]{../DataStructures/library.py}


\subsection{Результат выполнения программы}

\begin{figure}[H]
    \centering
    \includegraphics[width=0.8\textwidth]{./flowcharts/result_lib.png}
    \caption{Результат работы программы}
\end{figure}










\newpage
\section{Задача 2}

\subsection{Представления графа}

Необходимо представить граф:

\begin{figure}[H]
    \centering
    \includegraphics[width=0.8\textwidth]{./flowcharts/graph.png}
    \caption{Граф}
\end{figure}



Все четыре четыре способа способа представления 
графа можно реализовать на любом языке программирования 
используя 2-х или 3-х мерные массивы.


\begin{table}[!ht]
    \centering
    \caption{Матрица смежности}
    \begin{tabular}{|c|c|c|c|c|c|c|}
    \hline
        № & 1 & 2 & 3 & 4 & 5 & 6 \\ \hline
        1 & 0 & 7 & 9 & 0 & 0 & 14 \\ \hline
        2 & 7 & 0 & 10 & 15 & 0 & 0 \\ \hline
        3 & 9 & 10 & 0 & 11 & 0 & 2 \\ \hline
        4 & 0 & 15 & 11 & 0 & 6 & 0 \\ \hline
        5 & 0 & 0 & 0 & 6 & 0 & 9 \\ \hline
        6 & 14 & 0 & 2 & 0 & 9 & 0 \\ \hline
    \end{tabular}
    \label{Матрица смежности}
\end{table}



\begin{table}[!ht]
    \centering
    \caption{Матрица инцидентности}
    \begin{tabular}{|c|c|c|c|c|c|c|c|c|c|c|}
    \hline
        № & 1-2 & 1-3 & 1-6 & 2-3 & 2-4 & 3-4 & 3-5 & 3-6 & 4-5 & 5-6 \\ \hline
        1 & 1 & 1 & 1 & 0 & 0 & 0 & 0 & 0 & 0 & 0 \\ \hline
        2 & 1 & 0 & 0 & 1 & 1 & 0 & 0 & 0 & 0 & 0 \\ \hline
        3 & 0 & 1 & 0 & 1 & 0 & 1 & 0 & 1 & 0 & 0 \\ \hline
        4 & 0 & 0 & 0 & 0 & 1 & 1 & 0 & 0 & 1 & 0 \\ \hline
        5 & 0 & 0 & 0 & 0 & 0 & 0 & 1 & 0 & 1 & 1 \\ \hline
        0 & 0 & 1 & 0 & 0 & 0 & 0 & 0 & 1 & 0 & 1 \\ \hline
    \end{tabular}
    \label{Матрица инцидентности}
\end{table}



\begin{table}[!ht]
    \centering
    \caption{Списки смежности}
    \begin{tabular}{|c|c|}
    \hline
        Номер вершины & Смежные вершины \\ \hline
        1 & 2, 3, 6 \\ \hline
        2 & 1, 3, 4 \\ \hline
        3 & 1, 2, 4, 5 \\ \hline
        4 & 2, 3, 5 \\ \hline
        5 & 3, 4, 5 \\ \hline
        6 & 1, 3, 5 \\ \hline
    \end{tabular}
    \label{Списки смежности}
\end{table}



\begin{table}[!ht]
    \centering
    \caption{Списки рёбер}
    \begin{tabular}{|c|c|}
    \hline
        Номер ребра & Вершины, соединенные этим ребром \\ \hline
        1 & 1, 6 \\ \hline
        2 & 1, 3 \\ \hline
        3 & 1, 2 \\ \hline
        4 & 2, 3 \\ \hline
        5 & 2, 4 \\ \hline
        6 & 3, 4 \\ \hline
        7 & 3, 6 \\ \hline
        8 & 4, 5 \\ \hline
        9 & 5, 6 \\ \hline
    \end{tabular}
    \label{Списки рёбер}
\end{table}



\subsection{Алгоритм дейкстры}

\textbf{Словестное описание алгоритма: }

В простейшей реализации для хранения чисел \textit{d[i]} можно использовать массив чисел, а для хранения принадлежности элемента множеству U — массив булевых переменных.

В начале алгоритма расстояние для начальной вершины полагается равным нулю, а все остальные расстояния заполняются большим положительным числом (бо́льшим максимального возможного пути в графе). Массив флагов заполняется нулями. Затем запускается основной цикл.

На каждом шаге цикла мы ищем вершину $v$ с минимальным расстоянием и флагом равным нулю. Затем мы устанавливаем в ней флаг в 1 и проверяем все соседние с ней вершины $u$ Если в них (в $u$) расстояние больше, чем сумма расстояния до текущей вершины и длины ребра, то уменьшаем его. Цикл завершается, когда флаги всех вершин становятся равны 1, либо когда у всех вершин c флагом 0 $d[i]=\infty$. Последний случай возможен тогда и только тогда, когда граф G несвязный. 

\textbf{Блок схема алгоритма дейкстры}


\begin{figure}[H]
    \centering
    \includegraphics[width=0.8\textwidth]{./flowcharts/graph.png}
    \caption{Блок схема алгоритм Дейкстры}
\end{figure}


\textbf{Реализация на языке python}


\lstinputlisting[language = python]{../dijkstra.py}



\textbf{Результат выполнения программы}

\begin{figure}[H]
    \centering
    \includegraphics[width=0.8\textwidth]{./flowcharts/result.png}
    \caption{Результат работы программы}
\end{figure}


\textbf{Комплексность}

Комплексность данного алгоритма $O(N^2)$


\section{Ссылки}
Ссылка на весь код представленный в этом документе, а также исходный код самого документа вы можете найти по этой ссылке

\href{https://github.com/orenvadi/LAB3}{https://github.com/orenvadi/LAB3}



\end{document}
